%%% Question 6 - - - - - - - - - - - - - - - - - - - - - - - - - - - - - -
\tikzset{
    ->,
    node distance=3cm,
    every state/.style={thick, fill=gray!10},
    initial text=$ $,
}

\question This question is about type-level programming. For this question, you may assume that all of GHC's language extensions are available to you.
\begin{parts}
    
    \part Deterministic finite automata (DFAs) are finite state machines which can be used to determine whether a sequence of symbols has a given pattern. A diagram representing an example DFA is shown below:

    \begin{center}
    \begin{tikzpicture}
    \node[state, initial] (q0) {$q_0$};
    \node[state, right of=q0] (q1) {$q_1$};
    \node[state, accepting, right of=q1] (q2) {$q_2$};
    \node[state, below of=q1] (q3) {$q_3$};
    
    \draw [>=stealth] (q0) edge[above] node{0} (q1)
          (q0) edge[below] node{1} (q3)
          (q1) edge[loop above] node{0} (q1)
          (q1) edge[above] node{1} (q2)
          (q2) edge[below] node{0} (q3)
          (q2) edge[loop above] node{1} (q2)
          (q3) edge[loop below] node{0,1} (q3);
    \end{tikzpicture} 
    \end{center}

    The initial state of the machine is $q_0$. To check whether a word is ``accepted'' by a DFA, we simply need to follow the arrows from the initial state. This DFA works on strings of binary characters. For example, for ``001'', we would start in $q_0$, follow the arrow to $q_1$, then $q_1$ again, then to $q_2$. The inner circle for $q_2$ means that it is an \emph{accepting} state. If we end up in an accepting after processing all symbols in the input, the word is accepted by the DFA. In this case, ``001'' is accepted, but \emph{e.g.} ``0010'' would not be. In general, this DFA accepts all words comprised of one or more 0s followed by one or more 1s.

    \begin{subparts}
        \subpart[2] Define a type \haskellIn{State} which enumerates states of the DFA shown above and a type \haskellIn{Symbol} which enumerates symbols of the binary alphabet so that, for example, \haskellIn{Q0 :: State} and \haskellIn{ZERO :: Symbol}. \droppoints 
        
        \begin{solution}
            \emph{Application. 1 mark per definition.}
\begin{minted}{haskell}
data State = Q0 | Q1 | Q2 | Q3 
data Symbol = ZERO | ONE
\end{minted}
        \end{solution}
        
        \subpart[4] Define a closed type family whose kind signature is given by
        \vspace*{0.2cm}
\begin{minted}{haskell}
Delta :: State -> Symbol -> State
\end{minted}
\vspace*{0.2cm}
        and which implements transitions between states of the above DFA. For example, \haskellIn{Delta Q1 ONE} should reduce to \haskellIn{Q2}. \droppoints 
        
        \begin{solution}
            \emph{Application.}
\begin{minted}{haskell}
type family Delta (q :: State) (e :: Symbol) :: State where 
    Delta Q0 ONE  = Q3 
    Delta Q0 ZERO = Q1 
    Delta Q1 ONE  = Q2 
    Delta Q1 ZERO = Q1
    Delta Q2 ONE  = Q2 
    Delta Q2 ZERO = Q3 
    Delta Q3 ONE  = Q3 
    Delta Q3 ZERO = Q3
\end{minted}
        \end{solution}
    
        \subpart[4] With the help of \haskellIn{Delta}, define a closed type family whose kind signature is given by
        \vspace*{0.2cm}
\begin{minted}{haskell}
DeltaHat :: State -> [Symbol] -> State
\end{minted}
\vspace*{0.2cm}
        and which performs transitions for every symbol in a list of symbols, starting from a given initial state. For example, the type \haskellIn{DeltaHat Q0 '[ZERO, ZERO, ONE]} should reduce to \haskellIn{Q2}. \droppoints 
        
        \begin{solution}
            \emph{Application.}
\begin{minted}{haskell}
type family DeltaHat 
    (q :: State) (e :: [Symbol]) :: State where 
    DeltaHat q '[] = q 
    DeltaHat q (e ': es) = DeltaHat (Delta q e) es
\end{minted}
        \end{solution}
    
        \ifprintanswers \else \pagebreak \fi
    
        \subpart[8] Define a closed type family whose kind signature is given by 
        \vspace*{0.2cm}
\begin{minted}{haskell}
Accepts :: State -> [State] -> [Symbol] -> Bool
\end{minted}
\vspace*{0.2cm}
        which determines whether a DFA accepts a given word. The first argument represents the initial state of the DFA, the second represents the list of final states, and the third argument represents the input word. For example, \haskellIn{Accepts Q0 '[Q2] '[ZERO, ZERO, ONE]} should reduce to \haskellIn{True}.
        \droppoints 
        
        \begin{solution}
            \emph{Application.}
\begin{minted}{haskell}
type family Elem (x :: k) (xs :: [k]) :: Bool where 
    Elem x '[]       = False 
    Elem x (x ': xs) = True 
    Elem x (y ': xs) = Elem x xs

type family Accepts 
    (q :: State) (fs :: [State]) (w :: [Symbol]) :: Bool where 
    Accepts q fs w = Elem (DeltaHat q w) fs
\end{minted}
        \end{solution}
        
        \subpart[3] Define a proxy type for symbols and explain the need for proxy types in Haskell. \droppoints 
        
        \begin{solution}
            \emph{Application+Comprehension. 1 mark for definition, 2 marks for explanation.} Types in Haskell are erased during compilation once they have been checked and are therefore unavailable at run-time. As a result, functions only accept arguments whose types are of kind \haskellIn{*}. Proxy types are used to ``carry'' types of other kinds into the scope of a function's type.
            
\begin{minted}{haskell}
data SProxy (s :: Symbol) = SProxy
\end{minted}
        \end{solution}
        
        \subpart[4] Define a suitable type class and suitable type class instances to reify type-level symbols. \droppoints 
        
        \begin{solution}
            \emph{Application.} 
\begin{minted}{haskell}
class ReifySymbol (q :: Symbol) where 
    reifySymbol :: SProxy q -> Symbol

instance ReifySymbol ZERO where 
    reifySymbol _ = ZERO 

instance ReifySymbol ONE where 
    reifySymbol _ = ONE
\end{minted}
        \end{solution}
        
    \end{subparts}
\end{parts}
