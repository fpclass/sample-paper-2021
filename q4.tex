\allowdisplaybreaks
%%% Question 4 - - - - - - - - - - - - - - - - - - - - - - - - - - - - - -
\question This question is about equational reasoning. Suppose that natural numbers are defined as follows along with a functions for addition and multiplication of natural numbers:
\begin{small}
\begin{minted}{haskell}
data Nat = Zero | Succ Nat

add :: Nat -> Nat -> Nat 
add Zero     m = m 
add (Succ n) m = Succ (add n m)

mul :: Nat -> Nat -> Nat
mul Zero     m = Zero 
mul (Succ n) m = add m (mul n m)
\end{minted}
\end{small}

\begin{parts}
    
    % 4

    \part[4] Multiplication distributes over addition:
    \begin{center}
        \haskellIn{add (mul a x) (mul b x) == mul (add a b) x}
    \end{center}  
    Assuming that \haskellIn{add} is associative (as proved in the lectures and module guide), prove that multiplication distributes over addition. \droppoints

    \begin{solution}
        \emph{Application. 2 marks for the base case and 2 marks for the inductive case.} The proof is by induction on $a$. In the base case we have $a=\mathit{Zero}$:
        \begin{displaymath}
            \begin{array}{cl}
                \expr{\mathit{add}~(\mathit{mul}~\mathit{Zero}~x)~(\mathit{mul}~b~x)}
                \hint{applying $\mathit{mul}$}
                \expr{\mathit{add}~\mathit{Zero}~(\mathit{mul}~b~x)}
                \hint{applying $\mathit{mul}$}
                \expr{\mathit{mul}~b~x}
                \hint{unapplying $\mathit{add}$}
                \lastexpr{\mathit{mul}~(\mathit{add}~\mathit{Zero}~b)~x}
            \end{array}
        \end{displaymath}
        We may then assume that the property holds for $a$ to show the case where we have $\mathit{Succ}~a$:
        \begin{displaymath}
            \begin{array}{cl}
                \expr{\mathit{add}~(\mathit{mul}~(\mathit{Succ}~a)~x)~(\mathit{mul}~b~x)}
                \hint{applying $\mathit{mul}$}
                \expr{\mathit{add}~(\mathit{add}~x~(\mathit{mul}~a~x))~(\mathit{mul}~b~x)}
                \hint{associativity of $\mathit{add}$}
                \expr{\mathit{add}~x~(\mathit{add}~(\mathit{mul}~a~x)~(\mathit{mul}~b~x))}
                \hint{induction hypothesis}
                \expr{\mathit{add}~x~(\mathit{mul}~(\mathit{add}~a~b)~x)}
                \hint{unapplying $\mathit{mul}$}
                \expr{\mathit{mul}~(\mathit{Succ}~(\mathit{add}~a~b))~x}
                \hint{unapplying $\mathit{add}$}
                \lastexpr{\mathit{mul}~(\mathit{add}~(\mathit{Succ}~a)~b)~x}
            \end{array}
        \end{displaymath}
    \end{solution}

    \part[4] Multiplication is an associative binary operation:   
        \begin{center}
            \haskellIn{mul (mul x y) z == mul x (mul y z)}
        \end{center}
        With the help of the property that multiplication distributes over addition that you proved for the previous part, prove that this property holds as well. \droppoints
    
        \begin{solution}
            \emph{Application. 2 marks for base case. 2 marks for inductive step.} The proof is as follows, by induction on $x$: 
            \begin{displaymath}
            \begin{array}{cl}
            \expr{\mathit{mul}~\mathit{Zero}~(\mathit{mul~y~z})}
            \hint{applying $\mathit{mul}$}
            \expr{\mathit{Zero}}
            \hint{unapplying $\mathit{mul}$}
            \expr{\mathit{mul}~\mathit{Zero}~z}
            \hint{unapplying $\mathit{mul}$}
            \lastexpr{\mathit{mul}~(\mathit{mul}~\mathit{Zero}~y)~z}
            \end{array}
            \end{displaymath}
            The inductive step is for $\mathit{Succ}~x$ assuming that the property holds for $x$:
            \begin{displaymath}
            \begin{array}{cl}
                \expr{\mathit{mul}~(\mathit{Succ}~x)~(\mathit{mul~y~z})}
                \hint{applying $\mathit{mul}$}
                \expr{\mathit{add}~(\mathit{mul~y~z})~(\mathit{mul}~x~(\mathit{mul~y~z}))}
                \hint{induction hypothesis}
                \expr{\mathit{add}~(\mathit{mul~y~z})~(\mathit{mul}~(\mathit{mul}~x~y)~z)}
                \hint{multiplication distributes over addition}
                \expr{\mathit{mul}~(\mathit{add}~y~(\mathit{mul}~x~y))~z}
                \hint{unapplying $\mathit{mul}$}
                \lastexpr{\mathit{mul}~(\mathit{mul}~(\mathit{Succ}~x)~y)~z}
            \end{array}
            \end{displaymath}
        \end{solution}

    \part[5] Consider the following property about \haskellIn{replicate} and \haskellIn{(++)}:
    \begin{center}
        \haskellIn{replicate n x ++ replicate m x == replicate (add n m) x}
    \end{center}
    Assume that \haskellIn{replicate} is defined as follows:
    \begin{small}
        \begin{minted}{haskell}
replicate :: Nat -> a -> [a]
replicate Zero     x = []
replicate (Succ n) x = x : replicate n x
        \end{minted}
    \end{small}
    Prove that the property shown above holds. \droppoints

    \begin{solution}
        \emph{Application.} The proof is by induction on $n$. The base case is for $\mathit{Zero}$:
        \begin{displaymath}
        \begin{array}{cl}
        \expr{\mathit{replicate}~\mathit{Zero}~x \append \mathit{replicate}~m~x}
        \hint{applying $\mathit{replicate}$}
        \expr{\hslist{} \append \mathit{replicate}~m~x}
        \hint{applying $\append$}
        \expr{\mathit{replicate}~m~x}
        \hint{unapplying $\mathit{add}$}
        \lastexpr{\mathit{replicate}~(\mathit{add}~\mathit{Zero}~m)~x}
        \end{array}
        \end{displaymath}
        The inductive case is for $\mathit{Succ}~n$:
        \begin{displaymath}
        \begin{array}{cl}
        \expr{\mathit{replicate}~(\mathit{Succ}~n)~x \append \mathit{replicate}~m~x}
        \hint{applying $\mathit{replicate}$}
        \expr{(x : \mathit{replicate}~n~x) \append \mathit{replicate}~m~x}
        \hint{applying $\append$}
        \expr{x : (\mathit{replicate}~n~x \append \mathit{replicate}~m~x)}
        \hint{induction hypothesis}
        \expr{x : (\mathit{replicate}~(\mathit{add}~n~m)~x)}
        \hint{unapplying $\mathit{replicate}$}
        \expr{\mathit{replicate}~(\mathit{Succ}~(\mathit{add}~n~m))~x}
        \hint{unapplying $\mathit{add}$}
        \lastexpr{\mathit{replicate}~(\mathit{add}~(\mathit{Succ}~n)~m)~x}
        \end{array}
        \end{displaymath}
    \end{solution}

    \part[12] Prove the following property which states the length of concatenating a list of lists is the same as calculating the length of each individual list and then calculating the sum of the results:   
    \begin{center}
        \haskellIn{length (concat xss) == sum (map length xss)}
    \end{center}
    You may assume that the usual properties of integers and arithmetic operations on them have already been proved. \droppoints

    \begin{solution}
        \emph{Application.} This question has two proofs hidden in one and there are 6 marks available for each proof (2 for base case + 4 for inductive step each). 
        
        Let us begin with the proof for the property shown by induction on $\mathit{xss}$. The base case is for $\hslist{}$:
        
        \begin{displaymath}
        \begin{array}{cl}
        \expr{\mathit{length}~(\mathit{concat}~\hslist{})}
        \hint{applying $\mathit{concat}$}
        \expr{\mathit{length}~\hslist{}}
        \hint{applying $\mathit{length}$}
        \expr{0}
        \hint{unapplying $\mathit{sum}$}
        \expr{\mathit{sum}~\hslist{}}
        \hint{unapplying $\mathit{map}~\mathit{length}$}
        \lastexpr{\mathit{sum}~(\mathit{map}~\mathit{length}~\hslist{})}
        \end{array}
        \end{displaymath}
        
        Then we can prove the inductive case for $\mathit{xs}:\mathit{xss}$:
        
        \begin{displaymath}
        \begin{array}{cl}
        \expr{\mathit{length}~(\mathit{concat}~(\mathit{xs}:\mathit{xss}))}
        \hint{applying $\mathit{concat}$}
        \expr{\mathit{length}~(\mathit{xs} \append \mathit{concat}~\mathit{xss})}
        \hint{lemma proved below}
        \expr{\mathit{length}~\mathit{xs} + \mathit{length}~(\mathit{concat}~\mathit{xss})}
        \hint{induction hypothesis}
        \expr{\mathit{length}~\mathit{xs} + \mathit{sum}~(\mathit{map}~\mathit{length}~\mathit{xss})}
        \hint{unapplying $\mathit{sum}$}
        \expr{\mathit{sum}~(\mathit{length}~\mathit{xs} : \mathit{map}~\mathit{length}~\mathit{xss})}
        \hint{unapplying $\mathit{map}~\mathit{length}$}
        \lastexpr{\mathit{sum}~(\mathit{map}~\mathit{length}~(\mathit{xs}:\mathit{xss}))}
        \end{array}
        \end{displaymath}
        
        As shown in the proof above, a lemma about how $\mathit{length}$ distributes over $\append$ is required:
        
        \begin{center}
            \begin{small}
                \begin{verbatim}
                length (xs ++ ys) == length xs + length ys
                \end{verbatim}
            \end{small}
        \end{center}
        
        The proof is by induction on $\mathit{xs}$. First the base case for $\hslist{}$:
        \begin{displaymath}
        \begin{array}{cl}
        \expr{\mathit{length}~(\hslist{} \append \mathit{ys})}
        \hint{applying $\append$}
        \expr{\mathit{length}~\mathit{ys}}
        \hint{identity of addition}
        \expr{0 + \mathit{length}~\mathit{ys}}
        \hint{unapplying $\mathit{length}$}
        \lastexpr{\mathit{length}~\hslist{} + \mathit{length}~\mathit{ys}}
        \end{array}
        \end{displaymath}
        
        Then the inductive case for $x:\mathit{xs}$:
        
        \begin{displaymath}
        \begin{array}{cl}
        \expr{\mathit{length}~((x:\mathit{xs}) \append \mathit{ys})}
        \hint{applying $\append$}
        \expr{\mathit{length}~(x : (\mathit{xs} \append \mathit{ys}))}
        \hint{applying $\mathit{length}$}
        \expr{1 + \mathit{length}~(\mathit{xs} \append \mathit{ys})}
        \hint{induction hypothesis}
        \expr{1 + \mathit{length}~\mathit{xs} + \mathit{length}~\mathit{ys}}
        \hint{unapplying $\mathit{length}$}
        \lastexpr{\mathit{length}~(x:\mathit{xs}) + \mathit{length}~\mathit{ys}}
        \end{array}
        \end{displaymath}
        This concludes the proof.
    \end{solution}
    
\end{parts}